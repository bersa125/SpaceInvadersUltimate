\section{Conclusions}
As a final result of the work described by this document, our group comes out of the experience brought by the project with a very positive attitude and having expanded and improved their personal knowledge with respect to the languages and the programming methods used.\\
In particular the application of general patterns to a complex practical context gave us the ability to setting up in the best way all the individual assigned tasks that led us to compose the final product without overlapping each other during the work. Moreover, assuming this programming approach applied to Threejs/WebGL, we obtained a very good result in the performances of the running application instead of using a naive implementation, without suffering from relevant drops of \textit{FPS}(\textit{Frames Per Second}) or wastes of memory despite having the animations of different indipendent models on the screen.\\
The application runs smoothly of most type of GPUs, but increasing the number of spaceship that are animated in the scene to a very high value makes the scene slowing down also using very simple models, suggesting that very complex animations and/or a lot of simultaneus operations need a better hardware to make the software run with acceptable performances. We also registered under the execution on machines mounting as operating systems different Linux distributions a radical FPS drop in the Firefox browser as a possible result of a bad behaviour in the usage of the system's GPU drivers. Similar issues haven't been found in another type of main browsers as majority of tests have been taken under Chrome/Chromium browser.